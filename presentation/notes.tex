% Created 2019-08-28 mer. 16:01
% Intended LaTeX compiler: pdflatex
\documentclass[hangsection=false, titlepage=false, tocnp=false]{cleanreport}
\usepackage[utf8]{inputenc}
\usepackage[T1]{fontenc}
\usepackage{graphicx}
\usepackage{grffile}
\usepackage{longtable}
\usepackage{wrapfig}
\usepackage{rotating}
\usepackage[normalem]{ulem}
\usepackage{amsmath}
\usepackage{textcomp}
\usepackage{amssymb}
\usepackage{capt-of}
\usepackage{hyperref}
\usepackage[most]{tcolorbox}
\usepackage{siunitx}
\newcommand{\authorFirstName}{Thomas}
\newcommand{\authorLastName}{Dehaeze}
\newcommand{\authorEmail}{dehaeze.thomas@gmail.com}
\author{Dehaeze Thomas}
\date{2019-08-28}
\title{Complementary Filters Shaping Using \(\mathcal{H}_\infty\) Synthesis\\\medskip
\large Control System Working Group meeting}
\hypersetup{
 pdfauthor={Dehaeze Thomas},
 pdftitle={Complementary Filters Shaping Using \(\mathcal{H}_\infty\) Synthesis},
 pdfkeywords={complementary filters, h-infinity, sensor fusion},
 pdfsubject={Complementary Filters Shaping Using H-Infinity Synthesis. Presentation during a Control System Working Group Meeting at LIGO.},
 pdfcreator={Emacs 26.2 (Org mode 9.2.5)},
 pdflang={English}}
\begin{document}

\maketitle

\section{Title Slide - Introduction}
\label{sec:org2616a5c}
Hi everyone,

Thank you for the invitation.

My name is Thomas Dehaeze and I am a PhD student working in the Precision Mechatronic Laboratory in Belgium.

I will present you recent work that myself, my supervisor Christophe Collette and a post-doc colleague Mohit Verma have done about sensor fusion and the synthesis of complementary filters.

\section{Sensor Fusion Architecture - Noise Filtering}
\label{sec:org3776d94}
A typical sensor fusion architecture is represented here where the signal of two sensors measuring the same quantity \(x\) are filtered out by two filters \(H_1\) and \(H_2\) and then combined to give a estimate \(\hat{x}\) of \(x\).

Each sensor have different noise characteristics \(n_1\) and \(n_2\) and dynamics \(G_1\) and \(G_2\).

We consider that the two filters \(H_1\) and \(H_2\) are complementary, meaning that the sum of their transfer function is equal to one.

If we first consider that the sensor dynamics is perfectly known, we can inverse the sensor dynamics to obtain \(G_1 = G_2 = 1\). And the estimate \(\hat{x}\) is equal to \(x\) plus the noise of the individual sensors filtered out by the associated filter.

The Power Spectral Density of the super sensor noise then depends on both the PSD of \(n_1\) and \(n_2\), but also on the norms of the complementary filters.

Thus, it is usually wanted that the filters are designed such that the norm of \(H_1\) is small when \(n_1\) is larger than \(n_2\) and the norm of \(H_2\) is small when \(n_2\) is large than \(n_1\). We can then obtain a super sensor that has overall less noise than both individual sensors.

\section{Sensor Fusion Architecture - Robustness}
\label{sec:org75a0990}
However, in practical systems, the sensor dynamics is never exactly known and cannot be perfectly inverted.

We can represent this by a multiplicative uncertainty on the dynamics of each sensor where
\begin{itemize}
\item \(w\) is a weight that represents the magnitude of the uncertainty
\item \(\Delta\) can be any stable transfer function with its \(\mathcal{H}_\infty\) norm less than 1
\end{itemize}

The super sensor dynamics uncertainty depends both on the uncertainty weights and on the complementary filters.

This dynamic uncertainty is problematic as it introduces unwanted phase lag and will limit the attainable bandwidth.

In order to limit the super sensor dynamic uncertainty, \(H_1\) and \(H_2\) have to be designed such that:
\begin{itemize}
\item the norm of \(H_1\) is small when the uncertainty on sensor 1 is large
\item the norm of \(H_2\) is small when the uncertainty on sensor 2 is large
\end{itemize}

Doing so, it is possible to obtain a super sensor with less dynamic uncertainty than the individual sensors.
This could allow to increase the control bandwidth.

With these two simple examples, we see that the norm of the complementary filters have a huge impact on both the performance and the robustness of the sensor fusion architecture.

This is why we worked on the development of a synthesis method that permits to shape the norms of the complementary filters.

\section{Shaping of Complementary Filters using \(\mathcal{H}_\infty\) synthesis}
\label{sec:org4d70a51}
The design objective is thus to design two complementary filters \(H_1\) and \(H_2\) such that the norms of \(H_1\) and \(H_2\) are below some defined weights \(W_1\) and \(W_2\).

If apply the \(\mathcal{H}_\infty\) synthesis on the shown generalized plant that includes the two weights \(W_1\) and \(W_2\), the algorithm will find a stable filter \(H_2\) such that the \(\mathcal{H}_\infty\) norm between \(w\) and \([z_1,\ z_2]\) is less than 1.

By then defining \(H_1\) to be the complementary of \(H_2\), we see that this \(\mathcal{H}_\infty\) synthesis problem corresponds to the design objective.

\section{Validation of the proposed synthesis method}
\label{sec:orgd012b55}
We then used this synthesis method to design complementary filters.

We started with relatively simple complementary filters.
Here, the dashed curves are the inverse magnitude of the chosen weights that defines the maximum allowed norm of the complementary filters.
The solid curves represents the synthesized complementary filters using the \(\mathcal{H}_\infty\) synthesis.

\section{Complementary Filters Used at LIGO - Specifications}
\label{sec:orgc32f967}
We then wanted to validate this synthesis method for the design of more complex complementary filters.

We chose one pair of complementary filters that are designed in the PhD thesis of Hua and used at the LIGO.

The specifications on the norms of the filters are shown by the black dashed lines and the solid curves are the inverse magnitude of the designed weighting functions.

The weights are designed to be as close as possible to the specifications in order to not over constrain the synthesis problem.
Also, the order of the weights are kept reasonably small as the synthesized complementary filter will have an order equal to the sum of the weights order.

The weighting functions used are a custom designed 7th order transfer function for the high pass filter and a Type I chebyshev filter of order 20 for the low pass filter.

\section{\(\mathcal{H}_\infty\) Synthesis - Comparison with LIGO's FIR filters}
\label{sec:orgdb88b2c}
After synthesis, we obtain the complementary filters shown by the dashed curves which are of order 27.
The FIR filters of order 512 develop in the PhD thesis of Hua are also shown by the solid curves.
The filters are quite similar both in phase and in magnitude.

To summarize:
\begin{itemize}
\item the specifications in terms of super sensor noise and dynamic uncertainty can be expressed as upper bounds on the filter's norm
\item the \(\mathcal{H}_\infty\) synthesis that we developed allows to shape complementary filters quite easily. It works very well for both simple and complex shapes
\item It can be easily generalized to the synthesis of more that two complementary filters. You can check the paper where this is explained
\end{itemize}

If you are interested by this work, the paper and all the Matlab scripts that was used to obtain all these results are accessible in the link.
\end{document}