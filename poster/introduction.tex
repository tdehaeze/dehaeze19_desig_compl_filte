%%% Local Variables:
%%% TeX-master: "poster"
%%% End:

Complementary filters are used when two or more sensors are measuring the same quantity with different noise characteristics.
Unreliable frequencies of each sensor are filtered out and then \textbf{combined to form a super sensor giving a better estimate over a wider bandwidth}.
This technique is called \textbf{sensor fusion} and is used in \textbf{many applications} ranging from the attitude estimation of UAVs~\cite{zimmermann92_high_bandw_orien_measur_contr} to the isolation systems for the LIGO~\cite{matichard15_seism_isolat_advan_ligo}.

As the super sensor characteristics largely depend on the \textbf{complementary filter norms}, their proper design is of primary importance for sensor fusion.
Although many design methods of complementary filters have been proposed in the
literature~\cite{jensen13_basic_uas,hua04_polyp_fir_compl_filter_contr_system},
no simple method that allows to shape the norm of the complementary filters is
available. Such method is proposed here and is based on the $\mathcal{H}_\infty$ synthesis.