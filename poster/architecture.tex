Let's consider \textbf{two sensors} measuring the same physical quantity \(x\) with
dynamics \(G_1(s)\) and \(G_2(s)\), and with uncorrelated noise characteristics
\(n_1\) and \(n_2\).

\bigskip

The signals from both sensors are fed into two \textbf{complementary filters} \(H_1(s)\)
and \(H_2(s)\) and then combined to form a \textbf{super sensor} that yield an estimate \(\hat{x}\) of \(x\) (Fig.~\ref{fig:fusion_super_sensor}).
\begin{equation*}
  \hat{x} = \left(G_1 H_1 + G_2 H_2\right) x + H_1 n_1 + H_2 n_2
\end{equation*}

\begin{minipage}[t]{0.69\linewidth}
  The \textbf{complementary property} of \(H_1(s)\) and \(H_2(s)\) implies that their transfer function sum is equal to one at all frequencies:
\end{minipage}\hfill
\begin{minipage}[t]{0.29\linewidth}
  \vspace{-2em}
  \begin{equation*}
    \tcmbox{H_1(s) + H_2(s) = 1}
  \end{equation*}
\end{minipage}

%%% Local Variables:
%%% mode: latex
%%% TeX-master: "poster"
%%% End:
